% TEMPLATE for Usenix papers, specifically to meet requirements of
%  USENIX '05
% originally a template for producing IEEE-format articles using LaTeX.
%   written by Matthew Ward, CS Department, Worcester Polytechnic Institute.
% adapted by David Beazley for his excellent SWIG paper in Proceedings,
%   Tcl 96
% turned into a smartass generic template by De Clarke, with thanks to
%   both the above pioneers
% use at your own risk.  Complaints to /dev/null.
% make it two column with no page numbering, default is 10 point

% Munged by Fred Douglis <douglis@research.att.com> 10/97 to separate
% the .sty file from the LaTeX source template, so that people can
% more easily include the .sty file into an existing document.  Also
% changed to more closely follow the style guidelines as represented
% by the Word sample file. 

% Note that since 2010, USENIX does not require endnotes. If you want
% foot of page notes, don't include the endnotes package in the 
% usepackage command, below.

\documentclass[letterpaper,twocolumn,10pt]{article}
\usepackage{usenix,epsfig,endnotes}
\begin{document}

%don't want date printed
\date{}

%make title bold and 14 pt font (Latex default is non-bold, 16 pt)
\title{\Large \bf Saving pins and power with integrated voltage converters}

\author{
{\rm Michael Barrow}\\
mbarrow@eng.ucsd.edu\\
University of California, San Diego
%\and
%{\rm Second Name}\\
%Second Institution
}

\maketitle

% Use the following at camera-ready time to suppress page numbers.
% Comment it out when you first submit the paper for review.
\thispagestyle{empty}


\subsection*{Abstract}
Recently, demand for increased processor performance coupled with reducing power budget has been addressed using emerging parallel processors. Parallel computation is an energy efficient \textbf{(cite chandrakasan)} way of increasing performance but requires wider interconnect busses. At the die boundary, the consequence is that systems face an IO bottleneck. The connection between silicon and substrate ends the scope of Moores law in a system, with IO density of packages increasing at a slower rate than on chip.\\
Compounding this problem, addressing performance by increasing paralell units result in increasing energy density due to the end of Dennard scaling \textbf{(cite)}. Devices therefore require an increasing number of power pins, further limiting IO pin availability. \textbf{HOW ABOUT COMPELLING EXAMPLE?}
We examine integrated power converters in this context. A review of the literature suggests with further research this technique could address the IO bottleneck of future processors.% and power efficiency of future processors.         

\section{Introduction}

%As Moores law continues, Technology scaling gives an increasing number of building blocks to meet conflicting consumer demands of increasing performance and energy efficiency. However the turn right approach exposes architects to complications of inconsistent technology scaling rates. In particular, for IO bound applications, the amount of achievable parallelism in silicon is proportional to the number of IO pins. The  \\

%\begin{itemize}
%\item{\textbf{Motivation}:  crisis has been known for some ti}
%\item{Stakeholders}
%\item{Historical context}
%\item{Current state of art}
%\item{State of art limitations}
%\item{Content of doc}
%\end{itemize}

Against a backdrop of declining PC and growing smartphone sales, mobile devices drive demand for low power, high performance architectures. Emerging applications such as augmented reality, location aware services, high performance games and novel machine interfaces place an increasing demand for expanded IO capabilities and memory bandwidth from generation to generation of these devices. With the number of popular software stacks being lower than the number of major hardware vendors, device battery life is an important product differentiator. Research interest has therefore increased in power efficient architectures that can meet the IO and memory bandwidth requirements of the mobile segment.\\
%DROP THISDespite a drop in PC demand, server 
\indent package IO density is a well known problem to architects. Marbell et al~\cite{Marbell2011} review 130 hardware designs over 30 years and conclude the historical trend of package pin count and power pin count increase at an unsustainable rate. Promising solutions have been proposed, with Chang et al~\cite{Chang2010} identifying a subset of practical approaches in 2010. They conclude that voltage scaling as proposed by Dennard et al~\cite{Dennard1974} is feasible, but enabled by a sum of techniques in different disciplines.\\
At the architectural level, power converter integration is advised in the worst case, where challenges of sub-threshold leakage prevent further reduction of CMOS voltage and power density prevents further integration of IO intensive system blocks such as memory. The observation in~\cite{Chang2010} is that reducing CMOS voltage has an exponential impact on loss through the power delivery pins. By maintaining pin voltage to devices and reducing operating voltage of CMOS using an efficient on die DC DC voltage converter bridge, an IC has a higher effective power density without increasing the number of power pins.\\
\indent Research into integrated DC DC converters has been an active topic pre-empting~\cite{Chang2010} for other architectural benefits. As will be seen in detail in section \textbf{SECTION No}, DC DC converters typically feature bulky passive components. These increase the cost and footprint of mobile systems. Additionally, battery technology does not enjoy the improvement rate of CMOS, so operating voltage drops have opened a gulf between Battery supply voltage and IC input voltages. Techniques for extending battery life such as multiple voltage domains mean CPU's require multiple supply voltages. The symptoms of this are seen in the \textbf{Quallcomm snapdraggon 800 which features 9 off chip DC DC converters CITE}. Against this backdrop, the feasibility of integrated DC DC converters shown by Kurson et al~\cite{Kurson2003} invigorated research interest of integrated DC DC converters. A major motivation is removing the space and component costs of off chip converters \textbf{CITE A bunch of papers that feature this in abstracts}.\\
\indent State of the art in integrated DC DC converters is exemplified in Intel Haswell CPU's FIVR. Although a comparable DC DC converter exists in the literature~\cite{Sturcken2012}, to the authors knowledge this is the only example of an integrated DC DC converter supplying a general purpose CPU.\\
Intel's literature~\cite{Intel2010} suggests the FVIR is employed primarily to remove external DC DC converters and improve power efficiency. However a comparison of the pinout between Haswell \textbf{Cite} and the non FIVR predecessor \textbf{Ivybridge Cite} suggests that some power pins were saved too. This is unsurprising as Kurson notes in~\cite{Kurson2003} that power pins could be saved by integrated DC DC converters. As we note DC DC converters improve the IO density problem, we review the literature to asses the feasibility of optimizing power pins to IO pins by increasing supply voltage of an integrated DC DC converter greatly beyond typical CPU operating voltage.\\
\indent The rest of the document is ordered as follows:\\
\begin{itemize}
\item{The operation of the DC DC converter and its critical design parameters}
\item{Challenges associated with integrating DC DC Converters on die}
\item{Performance limitations of monolithic CMOS DC DC converters}
\item{Practicaly implementing integrated DC DC converters}
\item{Methods of optimizing IC pin out with integrated DC DC converters}
\item{Conclusions}
\end{itemize}
 

More fascinating text. Features\endnote{Remember to use endnotes, not footnotes!} galore, plethora of promises.\\

\section{The DC DC converter}

\textbf{Definition of the problem}\\
How it works, critical parameters

\subsection{Problems of an Integrated DC DC Converter}

\textbf{Definition of the system problem lives in}\\
History of research\\
System type 1(Switched capacitor AND HISTORY)\\
System type 2(Integrated buck AND HISTORY Kurson book)\\
Want to show how General CMOS components are made for this role and why they didn't work.\\

\subsection{Performance drawbacks of Baseline DC DC Converters}

\textbf{Taxonomy of problem (deeper drill and more specifics of the problem)}\\
what is wrong with the switches?\\
what is wrong with the control? \\
What is wrong with the capacitors?\\
What is wrong with the inductors?\\

\subsection{Solutions}

\textbf{System: well defined, class of problems: well defined, now we will see "fixed" systems}\\
\begin{itemize}
\item{Switches fixes paper list and description DONT USE EXOTIC TECH! THESE WERE NEVER USED ON INTEGRATED DESIGNS!!!}
\item{Control fixes paper list and description NOTE THAT THIS PROBLEM IS WELL UNDER CONTROL, CONCLUDE FROM THIS SECTION}
\item{Capacitor fixes paper list and description NOTE THAT THIS PROBLEM CANNOT BE "SOLVED" FROM A SC TOPOLOGY POINT OF VIEW AND INDUCTORS ARE "BETTER" IN TERMS OF ENERGY DENSITY AND LINE REGULATION}
\item{Inductor fixes paper list and description. NOTE THAT THIS IS "SOLVED" BUT FOR SEMI-INTEGRATED OR WHATEVER}
\end{itemize}

\section{Saving pins with DC DC converters}

\textbf{Solutions \& Stakeholders, why is there still work to be done?}\\
Need to define the problem of Signal to power pin ratio. We cannot optimize this today with DC DC converters, because the \@ CMOS voltage stepdown means we need to have a lot of CORE power pins.\\
Our "stakeholder" is high voltage stepdown. If we do it we could reduce \textbf{DEFINE}"CORE POWER PINS" to an amount desirable for \textbf{DEFINE} signal integrity\\

\subsection{The IO limit with integrated DC DC converters}

\textbf{Defines stakeholder specific problem}\\
Think about this, it has to be differentiated from the section above

\subsection{Solutions}

Here we have to talk about:\\
3 the micro transformer solution to this problem (it could be made promising) \\
2 the MIT solution to this problem (it does not look promising)\\
1 the disruptive technology of Si on Insulator, and how it could be paired with interposers\\
In the mobile segment Silicon interposers can be used to solve the IO problem as seen in [Ultra-high I/O Density Glass/Silicon Interposer for high Band-witdh Smart Mobile Application].\\
This interposer allows signal integrity to be maintained at low voltage from the interposer to memory. a High voltage from the substrate and DC DC converter integrated to the interposer allows power density to be met with fewer pins on the package. In this scenario, the package can be made smaller to allow denser integration, or the io bandwidth  can be bound to signal integrity requirements...\\


\textbf{SHOULD HAVE SOME SIMULATION TO CHECK INSIGHT ON 1, 2 and 3}\\

\subsection{Conclusions}

Basically conclude that given the solutions simulation we could reach a situation where IO pressure is lifted from core power perspective, but that isn't so great tbh. 
  


{\footnotesize \bibliographystyle{acm}
\bibliography{sample.bib}}


\theendnotes

\end{document}







